\documentclass{llncs}

% The following do not comply with LLNCS standards!
\setlength{\oddsidemargin}{0.7cm}
\setlength{\evensidemargin}{0.7cm}
\setlength{\textwidth}{15.0cm}
\setlength{\textheight}{22.9cm}

\usepackage{graphicx}

\pagestyle{plain}

\begin{document}

\thispagestyle{empty}

\title{A Heuristic Algorithm for the Design of High-Density Oligonucleotide Microarrays}
\author{S\'ergio A. de Carvalho Jr.\inst{1} \and Sven Rahmann\inst{2}}
\institute{
Graduiertenkolleg Bioinformatik, Bielefeld University, D-33594 Bielefeld, Germany. \email{Sergio.Carvalho@cebitec.uni-bielefeld.de}
\and Algorithms and Statistics for Systems Biology group, Genome Informatics,\\Faculty of Technology, Bielefeld University, D-33594 Bielefeld, Germany. \email{Sven.Rahmann@cebitec.uni-bielefeld.de}
}

\maketitle

\begin{abstract}
% \section{Motivation:}
The production of most commercial DNA probe arrays is based on a parallel chemical synthesis usually driven by a set of photolithographic masks. Due to the natural properties of light and the ever shrinking feature sizes, the arrangement of the probes on the chip and the order in which their nucleotides are synthesized play an important role on the quality of the final product. We address the issue of evaluating a design and review existing algorithms that minimize the risk unintended illumination. We also present a new heuristics, called Priority-driven Re-embedding.
% TODO add this?
% that proved to be fast and efficient even with the latest million-probe microarrays.
% \section{Results:}
% TODO add this?
% Our new algorithm outperforms the best known post-placement optimization achieving x-y\% further reduction of conflicts with a manageable increase in running time.
% TODO how much improvement??
% \section{Availability:}
\end{abstract}

% \noindent{\bf Keywords:} microarray layout, placement algorithms, border length minimization.

\section{Introduction}

A DNA microarray is a piece of glass or plastic on which single-stranded fragments of DNA, called \emph{probes} are affixed or synthesized. The chips produced by Affymetrix are considered the industry standard. They can contain more than one million spots (or \emph{features}) as small as 11 $\mu$m, with each spot accommodating several million copies of a probe. Probes are typically 25 nucleotides long and are synthesized in parallel, on the chip, in a series of repetitive steps. Each step appends the same nucleotide to probes of selected regions of the chip. Selection occurs by exposure to light with the help of a photolithographic mask that allows or obstructs the passage of light accordingly \cite{FODOR91}.

Formally, we have a set of probe sequences $\mathcal{P} = \{p_{1}, p_{2}, ... p_{n}\}$ that are produced by a series of masks $\mathcal{M} = (m_{1}, m_{2}, ... m_{\mu})$, where each mask $m_{i}$ induces the addition of a particular nucleotide $\nu_{i} \in \{A, C, G, T\}$ to a subset of $\mathcal{P}$. The \emph{nucleotide deposition sequence} $\mathcal{S} = \nu_{1} \nu_{2} \ldots \nu_{\mu}$ corresponding to the sequence of nucleotides added at each masking step is therefore a supersequence of all sequences of $\mathcal{P}$.

In general, a probe can be \emph{embedded} within $\mathcal{S}$ in several ways. We can think of an embedding of $p_{j}$ as a $\mu$-tuple $(e_{j,1}, e_{j,2}, ... e_{j,\mu})$ in which $e_{j,i}$ equals 1 if probe $p_{j}$ receives nucleotide $\nu_{i}$ (at step $i$), or 0 otherwise. We say that the embedding of $p_{j}$ is \emph{productive} at step $i$ when $e_{j,i} = 1$, or \emph{unproductive} when $e_{j,i} = 0$. Equivalently, given an arrangement of the probes on the chip, we can say that a spot $s$ is productive at step $i$ if mask $\mu_{i}$ allows the passage of light to $s$ (probes of $s$ are activated to chemical coupling with nucleotide $\nu_{i}$), or unproductive otherwise.

\section{Layout Evaluation}

\subsection{Border Length}

Due to diffraction of light or internal reflection, untargeted spots can sometimes be accidentally activated in a certain step, producing unpredicted probes that can compromise the results of an experiment. This issue was described by Fodor et al.\ \cite{FODOR91} who noted that the problem is more likely to occur near the borders between masked and unmasked spots. Feldman and Pevzner \cite{FELDMAN93} were the first to formally address the problem, but they only considered \emph{uniform arrays} (arrays containing all possible probes of a given length). Hannenhalli et al.\ \cite{HANNENHALLI02} were the first to work with \emph{assay} arrays with arbitrary probes. They formulated the Border Length Minimization Problem (BLMP), which aims at finding an arrangement of the probes together with their embeddings that minimizes the number of border conflicts during mask exposure steps.

Given a mask $\mu_{i}$ we compute its border length as the number of borders shared by masked (unproductive) and unmasked (productive) spots. The total border length of a given arrangement is the sum of border lengths over all masks.

\subsection{Conflict Index}

In \cite{KAHNG03}, Kahng et al.\ noted that the definition of border length did not take into account two simple yet important practical considerations: a) stray light might activate not only immediate neighbors but also probes that lie as far as three cells away from the targeted spot; and b) imperfections produced in the middle of a probe are more harmful than in its extremities. With these observations in mind, we define the conflict index $\kappa(s)$ of a spot $s$ whose probes of length~$\ell_{s}$ are synthesized in $\mu$~masking steps as follows. First we define a distance-dependent weighting function, $\delta(s,s',i)$, that accounts for observation a) above:

\begin{equation}
\label{eq:dist_weight} \delta(s,s',i) :=
        \left\{
                \begin{array}{ll}
                        0 & \mbox{if spot $s'$ is masked at step $i$}, \\
                        \frac{1}{(d(s,s'))^{2}} & \mbox{otherwise}, \\
                \end{array}
        \right.
\end{equation}
%%
where $d(s,s')$ is the Euclidian distance between spots $s$ and $s'$. We also use position-dependent weights as suggested in \cite{KAHNG03} to account for observation b):
%%
\begin{equation}
\label{eq:pos_mult} \omega(s,i) :=
        \left\{
                \begin{array}{ll}
                        0 & \mbox{if spot $s$ is unmasked at step $i$}, \\
                        1 + \log{\min(b_{s,i} + 1,\ell_{s} - b_{s,i} + 1)} & \mbox{otherwise}, \\
                \end{array}
        \right.
\end{equation}
%%
where $b_{s,i}$ denotes the number of nucleotides synthesized at spot $s$ up to and including step $i$. We now define the conflict index of a spot $s$ as

\begin{equation}
\label{eq:conf_idx} \kappa(s) := \sum_{i=1}^{\mu} \left( \omega(s,i) \sum_{s'} \delta(s,s',i) \right),
\end{equation}
%%
where $s'$ ranges over all spots near $s$ (in practice, only those inside a 7x7 grid centered in $s$).

It should be clear that our definition of conflict index is intimately connected to that of border length although the first estimates the risk of producing faulty probes in a given spot while the latter measures the quality of a mask.

TODO: elaborate a bit more about the relation between border length and conflict index. Explain how the choice of position-dependent or distance weighting function can influence the border lenght when optimizing the conflict index. Also, mention that the \cite{KAHNG03} suggested to use square root for the position-dependent and that we decided to use log so that the border length is not significantly increased.

\section{Optimum Single Probe Embedding (OSPE)}

Describe the formulation of \cite{KAHNG03} again (for completeness of exposition)?

Note that it can be used to minimize both border length and conflict indices.

\section{Previous Work}

Define placement and post-placement algorithms.

Focus on post-placement: described Batched Greedy, Chessboard and Sequential re-embedding algorithms.

\section{Priority-driven Re-embedding Algorithm}

In this section we present our new heuristic post-placement optimization algorithm, called Priority-driven Re-embedding. Our approach is based on the following observation. While some probes have an astonishing number of possible embeddings (up to several millions on a typical Affymetrix chip), others may have only a few or even only one valid embedding into the deposition sequence. Those with a greater number of embeddings can better adapt to their neighbors. Those with only a limited number of embeddings are less flexible and should be re-embedded first. The idea is that we can use these probes with less degree of freedom to \emph{drive} the re-embedding of their neighboring spots.

Our algorithm can be described as follows. First, we find a set of spots whose probes have only one or a limited number of embeddings. We mark these spots as \emph{finished} and add their non-empty immediate neighbors to a priority queue of \emph{pending spots} whose order is determined by the number of embeddings their corresponding probes have. Then, we retrieve (and remove) the first spot from the queue (the one whose probe has the least number of embeddings) and use the OSPE algorithm to re-embed its probe optimally. Finally, we mark the spot as \emph{finished} and add its immediate neighbors to the queue. The algorithm repeats until the queue is fully emptied. Whenever an empty spot is reached, it is immediately marked as finished and, of course, not added to the queue.

Sometimes the layout of a chip contains ``islands'' of probes surrounded by empty spots. In such cases, it is necessary to do a full scan on the chip looking for non-finished spots. If any is found, it is added to the queue and processed as described above.

When computing the optimum embedding with the SPOE algorithm, we can consider all neighbors of a spot or only those which have already been marked as finished. Our experiments showed that the latter approach is more efficient. Moreover, it also allows it to be run several times in succession, with further reduction of conflicts.

As with the Sequential re-embedding algorithm, the SPOE is used here as a sub-routine and can work as a border length or conflict index minimization function.

We have also explored two variants in the ordering of the priority queue. The first one retrieves probes with the greatest number of finished spots (instead of the least number of embeddings). The intuition is that these spots have a lesser degree of freedom since they have to adapt to a greater number of neighboring probes and, thus, should be re-embedded first. In particular, a spot with 4 finished immediate neighbors should be re-embedded 

Another possibility is to order the queue based on a combination of number of embeddings and number of finished neighbors. 

Another possibility: . Re-embedded their neighbors but giving priority to those that 

Present three variants: 1) priority based on the number of embeddings; 2) 

\section{Experimental Results}

The description of the sequential algorithm did not mention whether the new OSPE algorithm considered all neighboring probes or only those who have already been re-embedded. Thus, we have implemented two variants Also, was not described with full details

\footnotesize

\begin{thebibliography}{99}

\bibitem{FODOR91} Fodor, S., Read, J., Pirrung, M., Stryer, L., Lu, A., Solas, D.:
Light-directed, spatially addressable parallel chemical synthesis.
Science {\bfseries 251} (1991) 767--73

\bibitem{FELDMAN93} Feldman, W., Pevzner, P.:
Gray code masks for sequencing by hibridization.
Genomics {\bfseries23} (1994) 233--235

\bibitem{HANNENHALLI02} Hannenhalli, S., Hubell, E., Lipshutz, R., Pevzner, P.:
Combinatorial algorithms for design of DNA arrays.
Adv. Biochem. Eng. Biotechnol. {\bfseries77} (2002) 1--19

\bibitem{KAHNG03} Kahng, A. B., Mandoiu, I., Pevzner, P., Reda, S., Zelikovsky, A.:
Engineering a scalable placement heuristic for DNA probe arrays.
Proceedings of the Seventh Annual International Conference on Computational Molecular Biology (2003) 148--83

\bibitem{KAHNG03B} Kahng, A. B., Mandoiu, I., Reda, S., Xu, X., Zelikovsky, A.:
Evaluation of placement techniques for DNA probe array layout.
Proceedings of the IEEE/ACM International Conference on Computer-Aided Design (2003) 262--269
 
\end{thebibliography}

\end{document}